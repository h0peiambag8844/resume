%!TEX program = lualatex

\title{CV}
\documentclass[letterpaper,11pt]{article}

% Choose theme, e.g. black, RedViolet, ForestGreen, MidnightBlue
\def\theme{RedViolet}

\usepackage{simplecv}

\setlist[itemize]{noitemsep}

\boldname{Surname}{Name}{N.}

\begin{document}

% Heading
\headinginline{Kevin Liu}{
     Website: \website{kliu.io} \\
     Email: \email{kevin@kliu.io} \\
     LinkedIn: \linkedin{the-kevin-liu} \\
     GitHub: \github{kliu128}
}

\section{Education}

\outerlist{

     \entrybig
     {\textbf{Stanford University}}{Stanford, CA}
     {B.S. Computer Science}{2020\textendash 2024}

}

\section{Experience}

\outerlist{

     \entrybig
     {\textbf{Delphus}}{Boston, MA}
     {Co-founder \& CTO}{2018\textendash present}
     \innerlist{
          \entry{Majority committer to 20k+ LoC
               \href{https://gitlab.scintillating.us/scintillating/delphus}{open-source codebase} in React, TypeScript, Solidity. Designed system architecture and
               distributed integrations. Inventor on filed utility patent.}
          \entry{CMU NSF I-Corps alum, dlab VC finalist (top 7/200), Alchemist Accelerator
               finalist (top 1\%).}
          \entry{Presented at: MIT Bitcoin Expo, Babson College, HBS Tech/Healthcare Conf.
               2019\&20, MITEF Startup Spotlight, Roxbury Innovation Center.}
     }

     \entrybig
     {\textbf{OpenAI}}{San Francisco, CA}
     {Debater}{2020}
     \innerlist{
          \entry{Contracted honest debater testing OpenAI's Debate Protocol. Evaluated
               conceptual physics and mathematical problems to emulate potential future AI
               safety protocols.}
     }

}
\section{Skills}

Experienced: JavaScript (React, Redux), TypeScript, HTML/CSS, C\#,
Python, Java, Kotlin, Solidity, \LaTeX\\
Proficient: Rust, Bash, Ruby on Rails\\
Technologies: Kubernetes, Docker, LXC, KVM, Ansible, ML with
Tensorflow/PyTorch

\section{Projects}

\begin{itemize}[label=--]
     \item \textbf{Predicting income levels from satellite imagery, Stanford ACMLab}:
           Leveraged Pandas, numpy for data parsing. Trained 12-layer CNN with PyTorch on
           Google Colab.

     \item \textbf{Cortex}: Social proof, key management, and data authorization platform
           on Ethereum and NuCypher. Uses cryptographic signatures to validate domain name
           ownership. Made for ETHWaterloo 2019.

     \item \textbf{Home server administration}: Bare-metal Kubernetes cluster on Proxmox.
           Proficient with HP iLO, Ansible. Public services secured with firewalls and
           automated HTTPS.
\end{itemize}

\textit{See also \url{https://kliu.io/projects}.}

\section{Leadership}

\begin{itemize}[label=--]
     \item Exec Member, \textbf{Stanford Effective Altruism \& Existential Risks
                Initiative} (2020)
     \item Website Manager, \textbf{Stanford One for the World} (2020) --
           \href{https://onefortheworld.su.domains}{Designed website} to popularize
           global health and development donations.
     \item Senior Leader, \textbf{Resource Force} (2016-20) --
           \href{https://gasleaks.info}{Designed website} and performed data analysis
           to raise awareness for natural gas leaks. Presented to local and state
           government.
     \item Co-organizer \& Director of Logistics, \textbf{Hack3} (2019-2020) -- Managed and led online hackathon
           with 200+ attendees, \$50k in prizes.
     \item Co-captain, \textbf{AB Robotics} (2016-20) -- Wrote autonomous control code in
           Kotlin, using CV and Vuforia. Semifinalist at 2019 MassFTC State Championship.
\end{itemize}

\section{Achievements}

Major hackathons won: \textbf{ETHWaterloo 2019} (decentralized social identity
system), \textbf{MAHacks III} (scientific study management system; 1st), \textbf{LexHack}
(open-source Alexa equivalent; 2nd), \textbf{MIT Blueprint 2018} (incentivized
Q\&A system; 3rd).

\begin{flushright}
     \textit{Version 2021.01}
\end{flushright}

\end{document}
