\title{CV}
%
% tccv (two columns curriculum vitae) is a LaTeX class inspired by the template
% found at latextemplates.com by Alessandro Plasmati.
%
% Create by Nicola Fontana, the original files can be downloaded from:
% http://dev.entidi.com/p/tccv/
%
\documentclass[paper=letter]{tccv}
\usepackage[T1]{fontenc}
\usepackage[english]{babel}
\usepackage[super]{nth}
\newcommand{\myhref}[2]{\href{#1}{#2}\footnote{\url{#1}}}

\begin{document}

\part{Kevin Liu}

\personal
    [linkedin.com/in/the-kevin-liu] {Boxborough, MA 01719} {+1 (781) 315-6703}
    {kevin@kliu.io} {kliu.io}

\section{Education}

\begin{yearlist}
     \item{2020-2024}
          {Stanford University} {B.S. Computer Science}
\end{yearlist}

\section{Experience}

\begin{eventlist}

\item{2018 -- Present}
     {\myhref{https://delph.us}{\textbf{Delphus}}} {Co-founder \& CTO}
\end{eventlist}

\vspace{-0.75cm}

Majority committer to 20k+ LoC
\myhref{https://gitlab.scintillating.us/scintillating/delphus}{open-source codebase} in React, TypeScript, Solidity. Designed system architecture and
distributed integrations. Inventor on filed utility patent.

CMU NSF I-Corps alum, dlab VC finalist (top 7/200), Alchemist Accelerator
     finalist (top 1\%).

Presented at: MIT Bitcoin Expo, Babson College, HBS Tech/Healthcare Conf.
     2019\&20, MITEF Startup Spotlight, Roxbury Innovation Center.

\begin{eventlist}
    \item{2020 -- Present}
         {OpenAI}{Debater}

    \vspace{-0.3cm}
    Contracted honest debater testing OpenAI's Debate Protocol. Evaluated
    conceptual physics and mathematical problems to emulate potential future AI
    safety protocols.
\end{eventlist}

\section{Skills}

\textsc{Experienced}: JavaScript (React, Redux), TypeScript, HTML/CSS, C\#,
     Python, Java, Kotlin, Solidity, \LaTeX\\
\textsc{Proficient}: Rust, Bash, Ruby on Rails\\
\textsc{Technologies}: Kubernetes, Docker, LXC, KVM, Ansible, ML with
Tensorflow/PyTorch


\section{Projects}

\textbf{Predicting income levels from satellite imagery, Stanford ACMLab}:
Leveraged Pandas, numpy for data parsing. Trained 12-layer CNN with PyTorch on
Google Colab.

\textbf{Cortex}: Social proof, key management, and data authorization platform
on Ethereum and NuCypher. Uses cryptographic signatures to validate domain name
ownership. Made for ETHWaterloo 2019.

\textbf{Home server administration}: Bare-metal Kubernetes cluster on Proxmox.
Proficient with HP iLO, Ansible. Public services secured with firewalls and
automated HTTPS.

\textit{See also \url{https://kliu.io/projects}.}

\section{Leadership}

Exec Member, \textbf{Stanford Effective Altruism \& Existential Risks
Initiative} (2020)

Website Manager, \textbf{Stanford One for the World} (2020) --
\myhref{https://onefortheworld.su.domains}{Designed website} to popularize
global health and development donations.

Senior Leader, \textbf{Resource Force} (2016-20) --
    \myhref{https://gasleaks.info}{Designed website} and performed data analysis
    to raise awareness for natural gas leaks. Presented to local and state
    government.
    
Co-organizer \& Director of Logistics, \textbf{Hack3} (2019-2020) -- Managed and led online hackathon
    with 200+ attendees, \$50k in prizes.

Co-captain, \textbf{AB Robotics} (2016-20) -- Wrote autonomous control code in
Kotlin, using CV and Vuforia. Semifinalist at 2019 MassFTC State Championship.

\section{Achievements}

Major hackathons won: \textbf{ETHWaterloo 2019} (decentralized social identity
system), \textbf{MAHacks III} (scientific study management system; 1st),LexHack
(open-source Alexa equivalent; 2nd), \textbf{MIT Blueprint 2018} (incentivized
Q\&A system; 3rd).

% Align to bottom
\mbox{}
\vfill
\begin{flushright}
    \textit{Version 2020.11}
\end{flushright}

\end{document}
